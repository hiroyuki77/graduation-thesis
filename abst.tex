\setlength{\parindent}{12pt}
                                                                         

汎用オペレーティングシステムは様々な利用形態に対応するために巨大化・複雑化している.
しかし,クラウドコンピューティングの普及に伴い一台の仮想マシン上で一つのアプリケーションのみを実行することを想定する場合も増えてきている.
予備実験によると,HTTPサーバの典型的な挙動をシミュレーションするワークロードにおいて,86.1%のカーネルのコア部分のコードが利用されていない.
この結果から,多くのカーネル機能は利用されておらずメモリ使用量の増加に伴うオーバヘッドとなっており,
またセキュリティ問題が発生する可能性を高めてしまっていることがわかる.

本研究では汎用オペレーティングシステムを一つのアプリケーションに特化するように自動改変する手法を提案する.
アプリケーションを静的解析し,必要なカーネル機能を絞り込むことで,Linuxカーネルの必要な部分のみを残すことを方法とする.
自動改変によって,アプリケーションやそのバージョンごとにカーネルを作りなおす多大な労力が不要となる.
一方,汎用オペレーティングシステムの改変には,複雑な依存関係を持つ数万に及ぶカーネル関数から,最低限必要な関数のみを効率的に抽出する必要がある.
本研究は複数台の仮想マシンを用いて,必要な関数の抽出とコードの改変および動作検証の一連の流れを自動化する.

本研究の結果,汎用オペレーティングシステムの最小化を実現するための様々な課題や知見を得ている.
例えば,コードカバレッジから起動に必要な関数を抽出すると,コンパイラの最適化によって必要な関数を不必要と判断してしまうことや,インラインアセンブラ内の関数定義を誤って削除してしまうことなどが挙げられる.
